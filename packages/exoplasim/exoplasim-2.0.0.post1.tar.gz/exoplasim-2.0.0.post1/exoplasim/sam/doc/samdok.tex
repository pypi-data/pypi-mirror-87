\documentstyle[12pt,german,fleqn]{article}
\newfont{\fett}{cmssbx10 at 12pt}
\setlength{\unitlength}{1cm}
\setlength{\textwidth}{6.0in}
\setlength{\textheight}{9.6in}
\setlength{\topmargin}{-0.5in}
\setlength{\oddsidemargin}{0.25in}
\setlength{\intextsep}{0.0in}
\tolerance10000
\renewcommand{\baselinestretch}{1} 
\renewcommand{\topfraction}{.9} 
\renewcommand{\bottomfraction}{.9} 
\renewcommand{\textfraction}{.05} 
\renewcommand{\floatpagefraction}{.5} 
\setlength{\parskip}{\bigskipamount}
\setlength{\parindent}{0em} 
%\setlength{\mathindent}{0em} 
\setcounter{topnumber}{2} 
\setcounter{bottomnumber}{2} 
\renewcommand{\theequation}{%
              \mbox{\arabic{num}.\arabic{equation}\alph{alpheqn}}}%
\newcounter{num} 
\newcounter{alpheqn} 
\newcounter{alpheqn2} 
\newcounter{alpheqn3} 
\newcounter{saveeqn} 
\begin{document}  
{\pagestyle{empty}
\begin{center}

{\LARGE \bf Shallow water model SAM}

\vspace{1.in}

{\LARGE \bf Reference Manual}

\vspace{1.in}

{\LARGE \bf Version 1.0}

\vspace{1.in}

{\large ~T. Frisius, K. Fraedrich, E. Kirk and F. Lunkeit}



\end{center}


\newpage
}
\setcounter{page}{1}
\setcounter{num}{1}
\setcounter{alpheqn}{0}
\setcounter{equation}{0}

{\bf 1 Model equations}

The shallow water model SAM solves the shallow water equations on a
spherical planet with shallow orography. The model equations and the
numerical solution procedure will be described in the following subsections.

1.1 Continuity equation

\nopagebreak[4]

In a shallow water model it is assumed 
that the density $\rho$ is constant. Consequently, the continuity equation reduces to the
condition that the threedimensional flow is nondivergent.
In spherical coordinates the nondivergence condition reads
\begin{equation}
\left ( \frac  1 {r \cos \varphi}  \frac {\partial u} {\partial \lambda}
            +\frac  1 {r \cos \varphi}  \frac {\partial } {\partial \varphi} (\cos \varphi v)
                    +  \frac {\partial w} {\partial r}   + \frac {2 w} r      \right ) =0,
\end{equation}
In this equation $\lambda$, $\varphi$ and $r$ are the spherical coordinates longitude, latitude and 
radius, respectively and $u$, $v$ and $w$ denote the zonal, meridional and vertical velocity components, respectively. The last term on the left hand side of this equation can be neglected since it is assumed that
the shallow water system is contained within a thin spherical shell. Furthermore, the radius $r$ can be approximately
replaced be the mean planet radius $a$ where it appears in a factor.
Therefore, the continuity equation will be simplified to give
\begin{equation}
\left ( \frac  1 {a \cos \varphi}  \frac {\partial u} {\partial \lambda}
            +\frac  1 {a \cos \varphi}  \frac {\partial } {\partial \varphi} (\cos \varphi v)
                    +  \frac {\partial w} {\partial z}       \right ) =0,
\end{equation}
where $z=r-a$ denotes the height above the spherical surface having the radius $r=a$.

In a shallow water model it is assumed that the horizontal velocity components are vertically
independent. Therefore, the continuity equation can be vertically integrated over the whole
shallow water fluid. This gives
\begin{equation}
\frac {D h} {Dt} - \frac {D h_s} {Dt} + (h-h_s) \nabla_h \cdot {\bf v}_h = 0~,
\end{equation}
where $\nabla_h \cdot {\bf v}_h$ denotes the divergence of the horizontal flow, $h$
the height of the shallow water surface, $h_s$ the height of the shallow water bottom
and 
\[
\frac {D} {Dt}=\frac {\partial } {\partial t} + \frac u {a \cos {\varphi}} 
\frac {\partial} {\partial \lambda} + \frac v {a} 
\frac {\partial} {\partial \varphi} 
\]
is the individual time derivative operator.

1.2 Momentum equation

\nopagebreak[4]

The fluid of the shallow water model is treated as a nondivergent ideal fluid. Therefore,
the momentum equation is given by {\sc Euler}s' equation in a rotating coordinate frame:
\begin{equation}
\frac {\partial {\bf v}} {\partial t} + \nabla \left(\frac 1 2 {\bf v}^2 \right)
+ (\nabla \times {\bf v} + 2 \mbox{\boldmath $\Omega$}) \times {\bf v}
= - \frac 1 {\rho} \nabla p - \nabla \phi,
\end{equation}
where
$p$ is the pressure, $\phi$ the geopotential and  \mbox{\boldmath $\Omega$} the angular velocity vector of the planet rotation which is directed perpendicular to the
equatorial plane ($\varphi=0$).

The shallow water model describes 
flow phenomena having a vertical scale that is much smaller than its horizontal
scale. In this case the hydrostatic balance is satisfied to a high degree.
Therefore, it is assumed that the following hydrostatic balance equation holds
\begin{equation}
\frac {\partial p} {\partial z} = - \rho \frac {\partial \phi} {\partial z}
\end{equation}
In this equation $p$ denotes the pressure and
$\phi$ the geopotential.
Since density is constant, this equation can easily be
integrated in vertical direction over the whole fluid depth. This gives
\begin{equation}
p_s - \left . p \right |_{z=h} =  \rho ( \left . \phi \right |_{z=h} - \phi_s )~,
\end{equation}
where the index $s$ means that the variable is evaluated at the bottom of
the shallow water. 
For shallow water systems it is usally assumed that the pressure at the free upper surface vanishes. Furtheremore, for earth-like planets the gravitational attraction force in addition with the centrifugal force of the planet rotation give rise to  geopotential
isosurfaces that are nearly spherical. Therefore, it is assumed that the geopotential only
depends upon height $z$ and is, furthemore, proportional to height since the corresponding acceleration
is nearly constant with in a thin 
spherical shell. Consequently, surface pressure is related to the shallow water elevation
and surface orography by
\begin{equation}
p_s  =  g \rho (h - h_s)~,
\end{equation}
where $g$ denotes the gravitational acceleration.

Due to this equation the continuity equation (1.3) becomes a tendency equation for surface
pressure:
\begin{equation}
\frac {D p_s} {Dt}  + p_s \nabla_h \cdot {\bf v}_h  
= \frac {\partial p_s} {\partial t}  + \nabla_h \cdot ( {\bf v}_h p_s)=0 ~,
\end{equation}

The horizontal components of {\sc Euler}s equation are retained
with little approximation.
We obtain for the horizontal components of $(\nabla \times {\bf v} + 2 \mbox{\boldmath $\Omega$}) \times {\bf v}$:
{\setlength{\mathindent}{1em}
\begin{equation}
((\nabla \times {\bf v} + 2 \mbox{\boldmath $\Omega$}) \times {\bf v})_h
\end{equation}
\[
\hspace{3.cm}=\Omega (\zeta + 2 \sin \varphi) {\bf e}_z \times {\bf v}_h 
- \nabla_h \left(\frac 1 2 w ^2 \right) + \frac {{\bf v}_h w} r
+ 2 \Omega \cos {\varphi} w {\bf e}_{\lambda}~,
\]
}
where $\zeta$ is the nondimensional vertical component of the vorticity vector $\nabla \times {\bf v}$ 
(relative vorticity) and the index $h$ denotes that the vector has only horizontal components.
The last two terms on the right hand side can be neglected due to the small magnitude of
vertical velocity in the shallow water model.

Therefore, the  {\sc Euler}-equation for the horizontal velocity vector becomes:
\begin{equation}
\frac {\partial {\bf v}_h} {\partial t} + \nabla_h \left(\frac 1 2 {\bf v}_h^2 \right)
+ \Omega (\zeta + 2 \sin (\varphi)) {\bf e}_z \times {\bf v}_h
= - \frac 1 {\rho} \nabla_h p_s - \nabla \phi_s ~.
\end{equation}

1.3 Nondimensional model equations

\nopagebreak[4]

Applying the operators $\Omega^{-2} {\bf e}_z \nabla_h \times$ and $\Omega^{-2} \nabla_h \cdot$ gives the nondimensional vorticity and divergence equations, respectively
{\setlength{\mathindent}{1em}
\begin{equation}
 \frac {\partial \eta } {\partial \tau} 
= - \frac 1 {1 - \mu^2} \frac {\partial } {\partial \lambda}
\left ( \eta U \right ) - \frac {\partial } {\partial \mu}
\left (\eta V \right)~,
\end{equation}
\begin{equation}
 \frac {\partial D} {\partial \tau} 
=  \frac 1 {1 - \mu^2} \frac {\partial } {\partial \lambda}
\left(\eta V \right ) - \frac {\partial } {\partial \mu}
\left (\eta U\right)
- \Delta_h \left(\frac {U^2 + V^2} {2 (1 - \mu^2)}+ \Phi_s + {\rm Bu} P \right)  ~,
\end{equation}
}
Here, we have used the following notations
{\setlength{\mathindent}{0em}
\begin{eqnarray}
\nonumber \mu & =& \sin \varphi ~~~{\rm sine~ of~ latitude} \\ \nonumber
\tau & = & \Omega t~~~{\rm nondimensional~ time} \\ \nonumber
\eta  & = & \zeta + 2 \mu   ~~~{\rm nondimensional~absolute~vorticity} \\ \nonumber
D & = & \Omega^{-1} \nabla_h \cdot {\bf v}_h~~~{\rm nondimensional~ divergence} \\ \nonumber  (U,V) & =& \cos\varphi (\Omega a)^{-1} (u,v)~~~{\rm nondimensional~ velocity~ components~multiplied~by~} \cos{\varphi}  \\ \nonumber 
P & = & \frac  {p_s-p_0} {p_0}~~~{\rm nondimensional~ surface~ pressure~ anomaly} \\ \nonumber  
\Phi_s & = & \frac {\phi_s} {\Omega^2 a^2}~~~{\rm nondimensional~surface~geopotential}\\ \nonumber
{\rm Bu} & = & \frac {p_0} {\Omega^2 a^2  \rho}~~~{\rm {\sc Burger} ~ number} \\ \nonumber
\Delta_h & = &  \displaystyle \frac 1 {1- \mu^2}  \frac {\partial^2 } {\partial \lambda^2} + \frac {\partial } {\partial \mu} (1 - \mu^2) 
\frac {\partial } {\partial \mu} ~~~{\rm nondimensional~{\sc Laplace}-operator}
\end{eqnarray}
} 

Similarly, we obtain for the nondimensional surface pressure equation:
\begin{equation}
 \frac {\partial P} {\partial \tau} 
= - \frac 1 {1 - \mu^2} \frac {\partial } {\partial \lambda}
\left(U P \right ) - \frac {\partial } {\partial \mu}
\left (V P \right)  - D~,
\end{equation}

It remains to compute the horizontal velocity components from vorticity and
divergence fields. This can be done with the {\sc Helmholtz} decomposition of the velocity vector field:
\begin{equation}
{\bf v}_h=\Omega a^ 2 \left ( \nabla \chi +  \nabla \times {\bf A} \right)~.
\end{equation}
where $\chi$ is a nondimensional velocity potential and ${\bf A}$ a nondimensional vector stream potential. Because only
the horizontal velocity field is related to vorticity and divergence, it is sufficient to
restrict the vector potential  to a single vertical component, the so called streamfunction $\psi$.
Therefore:
\begin{equation}
U  = \displaystyle  - ( 1 - \mu^2) \frac {\partial \psi} {\partial \mu} +  \frac {\partial \chi} {\partial \lambda}~,~~~~
V  = \displaystyle \frac {\partial \psi} {\partial \lambda} +  ( 1 - \mu^2) \frac {\partial \chi} {\partial \mu}~.
\end{equation}
Streamfunction and velocity potential are related to vorticity and divergence via 
{\sc Poisson}-equations:
\begin{equation}
\eta= 2 \mu + \Delta_h \psi~,~~~~D  
= \Delta_h \chi ~.
\end{equation}

Equation (1.11)-(1.13), (1.15) and (1.16) form the governing equations of the shallow water model.
They are solved with a semispectral method which is described in some detail in the next subsections.

{1.4 Spectral representation of the model equations}

\nopagebreak[4]

All fields are projected onto a new base, namely, the so-called spherical harmonics $Y_n^m(\lambda,\mu)$. This gives for a field function $Q(\lambda,\mu,\tau)$:
\begin{equation}
Q(\lambda, \mu,\tau)=\sum_{m=-\infty}^{\infty} \sum_{n=|m|}^{\infty} Q_n^m(\tau) Y_n^m(\lambda,\mu)~.
\end{equation}
The spherical harmonics are defined by
\begin{equation}
Y^m_n(\lambda,\mu)=
\sqrt{\frac {2 n + 1} 2  \frac {(n-m)!} {(n+m)!}} \frac {(1- \mu^2)^{\frac m 2}} {2^n n!}
 \frac {\partial^{m+n}} {\partial \mu^{m+n}} \left [  (\mu^2 - 1)^n \right ]
e^{i m \lambda}~.
\end{equation}
Since $Q$ must be a real function we have the constraint $Q_n^m=Q_n^{-m*}$. Therefore, it is sufficient
to calculate only the coefficients with $m \ge0$ and the following representation is equivalent to (1.19):
\begin{equation}
Q(\lambda, \mu,\tau)=\sum_{n=1}^{\infty} Q_n^0(\tau) Y_n^0(\mu) + 2 \sum_{m=1}^{\infty} \sum_{n=m}^{\infty} {\rm Re} \left (Q_n^m(\tau) Y_n^m(\lambda,\mu) \right )~.
\end{equation}
The spherical harmonics form an orthogonal base with respect to the scalar product
\begin{equation}
\left < Q_1 | Q_2  \right > :=  \frac 1 {2 \pi} \int_{-1}^{1} \int_{0}^{2\pi} Q_1^*(\lambda,\mu) Q_2({\lambda,\mu}) d\lambda d\mu~
\end{equation}
Therefore:
\begin{equation}
\left < Y_n^m | Y_{n^{\prime}}^{m^{\prime}}  \right > = \delta_{m,m^{\prime}} \delta_{n,n^{\prime}}
\end{equation}
Furthermore, the spherical harmonics are eigenfunctions of the nondimensional
 {\sc Laplace}-operator $\Delta_h$:
\begin{equation}
\Delta_h Y_n^m=- n (n+1) Y_n^m~.
\end{equation}
Further useful relations are:
\begin{equation}
\frac {\partial Y^m_n} {\partial \lambda} 
=  i m   Y^m_n~,
\end{equation}
\begin{equation}
(1 - \mu^2) \frac {\partial } {\partial \mu} Y^m_n
=  n  d_{m,n+1} Y^m_{n+1} -  (n+1) d_{m,n} Y^m_{n-1}~, 
\end{equation}
where
\[
d_{m,n}=\sqrt{ \frac {n^2 - m^2} {4 n^2 -1} }~.
\]
With (1.22) - (1.24) the governing model equations can be converted into component form after applying the scalar product
$<Y^m_n|..>$:
{\setlength{\mathindent}{1em}
\begin{equation}
\frac {d \eta_n^m} {d\tau}=  -\left < Y^m_n ~|~   \frac {im\eta U} {1-\mu^2}\right > 
- \left < Y^m_n ~|~ \frac {\partial (\eta V)} {\partial \mu} \right >~,
\end{equation}
\begin{eqnarray}
\frac {d D_n^m} {d\tau} & = &   \left < Y^m_n ~|~  \frac {im \eta V} {1-\mu^2} \right >
- \left < Y^m_n ~|~ \frac {\partial (\eta U)} {\partial \mu} \right > \\ \nonumber
& & + n(n+1)
\left < Y^m_n ~|~ \frac 1  2 \frac  {U^2+V^2}  {1-\mu^2}  \right > + n(n+1) \left [ {\rm Bu} P_n^m + {\Phi_s}^m_n \right ]~,
\end{eqnarray}
\begin{equation}
\frac {d P_n^m} {d\tau}=  -\left < Y^m_n ~|~  \frac {im P U} {1-\mu^2} \right > 
- \left < Y^m_n ~|~ \frac {\partial (P V)} {\partial \mu} \right  > - D_n^m~,
\end{equation}
\begin{equation}
 U^m_n   =    (n - 1) d_{m,n}  \psi^m_n - (n+2) d_{m,n+1}  \psi^m_{n+1} + i m~ \chi^m_n~,
\end{equation}
\begin{equation}
  V^m_n  =   i m ~ \psi^m_n - (q - 1) d_{m,n}  \chi^m_{n-1}
  + (n+2) d_{m,n+1}  \chi^m_{n+1}~, 
\end{equation}
\begin{equation}
 \psi^m_n   =  - \frac 1 {n(n+1)} \left(\eta^m_n - \sqrt {\frac 8 3}~
 \delta_{m,0}\delta_{n,1}\right) ~, 
\end{equation}
\begin{equation}
 \chi^m_n   =  - \frac 1 {n(n+1)} D^m_n~.
\end{equation}
}

1.4 Numerical solution technique

\nopagebreak[4]

An approximative solution of the model is determined by 
time integration of the truncated spectral equations. In the truncated model
the first summation operator
in (1.17) runs only from $-M$ to $M$. This gives a triangular truncation
T$M$.

The terms in brackets of Eqs.(1.25)-(1.27) are nonlinear. The computation of these scalar products is very complicated and
consumes a lot of computer time.
Therefore, these terms are evaluated in the numerical model
 with the so-called spectral transform method [Orszag (1970)
and Eliasen et al. (1970)]. This method uses an auxiliary grid in the physical space where point values of the
dependent variables are computed. The auxiliary grid in the physical space ({\sc Gauss}ian grid) is defined
by $M_g$ equally spaced longitudes and $J_g$ {\sc Gauss}ian latitudes.
For the transformation from gridpoint space into spectral space the spectral coefficients are obtained by
{\sc Gauss}ian quadrature of the {\sc Fourier} coefficients. With this method bilinear products are
calculated without error when $M_g \ge 3M+1$ and $J_g \ge (3M+1)/2$. Since only bilinear products
appear as nonlinearities in the equations the spectral transform method is equivalent to the analytical
determination of the nonlinear products using interaction coefficients.

Time integration is done with the  semi-implicit time stepping procedure. This has the advantage that for a stable integration 
the time-step $\Delta \tau$ can be larger than the period of high-frequency inertial gravity modes. 
The time derivatives are approximated by a centered difference so that 
\begin{equation}
\frac {d Q} {d \tau} \approx \delta_t Q := \frac {Q(\tau+\Delta \tau) - Q(\tau+\Delta \tau)} {2 \Delta \tau}~.
\end{equation}
The tendency is splitted into an explicit and an implicit part. The explicit part is related to the sum of all nonlinear terms
and is evaluated at the present time  $\tau$. The linear terms form the implicit part. They only occur in the divergence and pressure
equation and enter the equations as a time mean of time step $\tau +\Delta \tau$ and $\tau - \Delta \tau$. Altogether, the finite-difference form of the prognostic model equations reads:
{\setlength{\mathindent}{1em}
\begin{equation}
\delta_t \eta_n^m = 
{\cal N}_{\eta}(\tau)~,
\end{equation}
\begin{equation}
\delta_t D_n^m  = {\cal N}_{D}(\tau) 
+ {\rm Bu}\frac {n(n+1)} {2} [P_n^m(\tau-\Delta \tau) + P_n^m(\tau+\Delta \tau)
+ 2 {\Phi_s}^m_n]
\end{equation}
\begin{equation}
\delta_t P_n^m =  {\cal N}_{P}(\tau) 
- \frac 1 2 [D_n^m(\tau-\Delta \tau) + D_n^m(\tau+\Delta \tau)]~,
\end{equation}
where ${\cal N}_{\eta}$, ${\cal N}_{D}$ and  ${\cal N}_{P}$ denote the
respective nonlinear tendency terms. Eliminating $P_n^m(\tau+\Delta \tau)$ in (1.35) gives
\begin{equation}
\left [1 + \Delta \tau^2 {\rm Bu}~{n(n+1)}  \right ] \overline{D_n^m}^{\tau} 
\end{equation}
\[ \hspace{0.1in}
= 
 D_n^m(\tau - \Delta \tau) 
+  \Delta \tau \left \{ {\cal N}_{D}(\tau) 
+ {\rm Bu}~{n(n+1)}   \left [ 
P_n^m(\tau - \Delta \tau) + {\Phi_s}^m_n + \Delta \tau {\cal N}_P(\tau) \right ] \right \}~,
\]
}
where $\overline{D_n^m}^{\tau}= [D_n^m(\tau +� \Delta \tau)+D_n^m(\tau - \Delta \tau)]/2$.

The time averaged value $\overline{D_n^m}^{\tau}$ can be easily calculated from equation (1.36).
With this value it is straightforward to determine the future values $D^m_n(\tau + \Delta \tau)$
and $P^m_n(\tau + \Delta \tau)$.

For noise reduction a {\sc Robert}/{\sc Asselin}-Filter [Haltiner and Williams
(1982)]  is applied 
at every timestep to a spectral coefficient $Q_n^m$ as follows:
\begin{equation}
Q^m_n(\tau)=(1-2 \gamma)Q^m_n(\tau) 
+ \gamma [Q^m_n(\tau+\Delta \tau)+Q^m_n(\tau-\Delta \tau)]~.
\end{equation}
In the model SAM $\gamma$ takes the value 0.02 by default.

1.5 Hyperdiffusion

In a turbulent flow regime a cascade from large scales to the dissipative range 
of the wavenumber spectrum  takes place. The model cannot resolve such small scales.
Therefore, kinetic energy dissipation must occur at resolvable scales in the model without
affecting the large scale too much. This is achieved by introducing hyperdiffusion by
adding the terms
\begin{equation}
{\cal H}_{\eta~n}^m= K \left [ - n(n+1) \right ]^{n_h}    \left(\eta^m_n - \sqrt {\frac 8 3}~
 \delta_{m,0}\delta_{n,1}\right)
\end{equation}
and
\begin{equation}
{\cal H}_{D~n}^m= K \left [ - n(n+1) \right ]^{n_h}   D^m_n
\end{equation}
to the vorticity and divergence equations, respectively. $n_h$ denotes the order of 
hyperdiffusion. Without additional terms the corresponding vorticity or divergence component
is damped with the dimensional $e$-folding timescale
\begin{equation}
\tau_H= \left \{ \Omega K [n (n+1)]^{n_h} \right \}^{-1} ~.
\end{equation}
In the model $\tau_H$ at the wavenumber where the expansion is truncated will be prescribed.
Therefore, the coefficient $K$ becomes:
\begin{equation}
K= \left \{ \Omega \tau_H [M (M+1)]^{n_h} \right \}^{-1}~.
\end{equation}

1.6 Parameterization of baroclinic and dissipative processes

To perform simulations with baroclinic forcing and dissipative damping additional terms must
be added to the spectral vorticity and divergence equations.
These are given by
\begin{eqnarray}
\left . \frac {d \eta_n^m} {d\tau} \right |^P  & = & \frac {\eta_{ne}^m -\eta_n^m } {\tau^m_{nR}} 
+ S^m_n (\tau)
- \frac {\eta_n^m - \sqrt{8/3} \delta_{m,0} \delta_{n,1} } {\tau_F} \\ \nonumber
& & - A_{h} \left ( n^2 + n  - 2 \right )  
\left (\eta_n^m - \sqrt{8/3} \delta_{m,0} \delta_{n,1} \right ) ~, 
\end{eqnarray}
\begin{equation}
\left . \frac {d D_n^m} {d\tau}\right |^P  =  - \frac {D_n^m  } {\tau_F} -  A_{h}   { n (n+1)}  D^m_n ~,
\end{equation}
where $\eta_{ne}^m$ is the equilibrium state of the climatological baroclinic forcing, $\tau^m_{nR}$ the timescale of the
climatological forcing (depends upon wavenumbers), $S^m_n(\tau)$ the time dependent baroclinic forcing (stochastic),
$\tau_F$ the timescale of Rayleigh friction and $A_h$ the horizontal  momentum exchange coefficient.


{\bf 2 User guide}

\nopagebreak[4]

SAM can like PUMA and PlaSim be compiled and started with the model starter program most.x.
SAM also supports the Graphical User Interface (GUI) developed for PUMA and PlaSim.
In the code several initial conditions are specified with the integer parameter iexp (1-13).
The various experiments are documented in the initialization subroutine initfd. Further
initial conditions can be added at discretion. The model resolution has to be specified with the
namelist {\it resolution\_namelist}. It contains the parameter NLAT that denotes the number of
latitudes. NLAT should be represented by $2^{n}$ with $n$ being an integer to ensure
optimal parallelization and consistency with the fast Fourier transform.

The namelist for SAM {\it somnamelist} contains the following parameters

\begin{tabular}{|l|c|l|}
\hline
Parameter & Default & Meaning\\
\hline
KICK     & 1        & Initial noise (not implemented in the current version)\\
NAFTER   & 12       & Output after intervals of NAFTER timesteps\\
NCOEFF   & 0        & Number of spectral modes to be print in wrspam\\ 
NDEL     & 6        & Order of hyperdiffusion\\
NDIAG    & 12       & Parameter to determine that ASCII diagnostics \\ 
         &          & are written after intervals of NDIAG timesteps\\
NKITS    &  3       & Number of initial timesteps\\
NRUN     &  0       & Number of timesteps to run - 0: use nyears and\\
NWSPINI  &  1       & 1: Write initial sp(:) to file puma\_sp\_ini\\
NOUTPUT  &  1       & 1: Write model output to file sam\_output\\
NSTEP    &  0       & Current timestep\\
NSTOP    &  0       & Stop step - 0: compute from nyears and nmonths\\
NTSPD    &  1       & Number of timesteps per day\\
NCU      &  0       & Check unit (for debug output only)\\
NGUI     &  0       & 1: Run with GUI\\
NGUIDBG  &  0       & 1: Switch on GUI debug output\\
NYEARS   &  0       & Simulation time in years\\
DISS     &  0.25    & Hyper diffusion time scale [days]\\
NRUIDO   &  0       & 1: Add noise on every time step\\
DISP     &  0.0     & Noise amplitude for nruido = 1\\
AH       &  0.0     & Horizontal momentum exchange coefficient [m$^2$/s]\\
RESTIM   &  0.0     & Timescale for climatological forcing [days] (0.0$\rightarrow$ no forcing)\\
NDL      &  0       & 1: Print spectral vorticity and divergence modes\\
ROTSPD   &  0.      & Rotations per day  (should be set as in PlaSim)\\
TFRC     &  0.      & Rayleigh friction timescale in days (0.0 $\rightarrow$  no friction)\\
LLID     & .false.  & Switch for introducing a rigid lid\\
LEQUIV   & .false.  & Switch for equivalent barotropic model\\
LBAL     & .false.  & Switch for balanced initial state\\
IEXP     & 4        & Number of experiment (1-13, see initfd)\\
MMAX     & 0        & Maximum zonal waven. for climat. forcing     \\
NMAX     & 42       & Maximum total waven. for climat. forcing     \\
\hline
\end{tabular}

Other parameters can be set in the module pumamod of the code. These are:
\begin{tabular}{|l|c|l|}
\hline
Parameter & Default & Meaning\\
\hline
GA       & 9.81       & Gravity acceleration [m/s$^2$]\\
PLARAD   & 6371000.0  & Radius of the sphere [m]\\
WW       & 0.00007292 & Angular velocity of the rotating sphere [1/s]\\
RHO0     & 1.         & Density of the shallow water [kg/m$^3$]\\ 
HS       & 12000.     & Height of the shallow water [m]\\
F0       & 1.e-4      & Mean Coriolis parameter [1/s] (only LEQUIV=.true.)\\
\hline
\end{tabular}

SAM writes output into the file {\it sam\_output}. It contains
spectral coefficients of i) surface geopotential 
(code 129), ii) logarithm of surface pressue (code 152),
iii) divergence (code 155) and iv) relative vorticity (code 138). 
The pumaburner 
can process the file  {\it sam\_output} for further diagnostics.

\end{document} 


