\begin{thebibliography}{}

\bibitem[Albertson and Kiely (2001)]{albertson2001}
 Albertson, J. D., G. Kiely, On the structure of soil moisture time series
 in the context of land surface models, {\it Journal of Hydrology},
 {\bf 243}, 101--119, 2001.

\bibitem[Apel (1987)]{apel1987}
 Apel, J. R.,
 Principles of Ocean Physics,
 {\em Academic Press},  Int. Geophys. Ser., {\bf 38}, 1987.

\bibitem[Beckmann and Birnbaum (2001)]{beckmann2001}
 Beckmann, A. and Birnbaum, G.,
 Cryosphere~: Coupled Sea Ice - Ocean Models,
 {\em Academic Press}, Encyclopedia of Ocean Sciences, 2001.

\bibitem[Betts and Ball (1997)]{betts1997}Betts, A. K. and J. H. Ball,
 Albedo over the boreal forest, {\it Journal of Geophysical Research},
 {bf 102}(D24), 28901-28909, 1997. 

\bibitem[Birnbaum (1998)]{birnbaum1998}
 Numerical modelling of the interaction between atmosphere and sea
 ice in the Arctic marginal ice zone,
 {\em Alfred Wegener Institute for Polar and Marine Research},
 PhD-thesis, 1998.

\bibitem[Bliss et al. (1981)]{bliss1981}
 Bliss LC, Heal OW, Moore JJ (eds) (1981) Tundra ecosystems:
 a comparative analysis. Cambridge University Press, Cambridge.

\bibitem[Buizza et~al. (1999)]{buizza1999}
 Buizza, R., Miller, M. and Palmer, T. N.,  Stochastic
 representation of model uncertainties in the ECMWF Ensemble
 Prediction System, {\em Q. J. R. Meteorol. Soc.}, {\bf 125},
 2887--2908, 1999.

\bibitem[Brooks et al. (1997)]{brooks1997}
 Brooks JR, Flanagan LB, Varney GT, Ehleringer JR, Vertical gradients
 of photosynthetic gas exchange and refixation of respired CO2 withini
 boreal forest canopies, {\it Tree Physiol},{\bf 17}, 1--12, 1997.

\bibitem[Bunce (2005)]{bunce2005}
 Bunce, J. A., What is the usual internal carbon dioxide concentration
 in C4 species under midday field conditions? {\it Photosynthetica},
 {\bf 43}, 603-608, 2005.

\bibitem[Cattle and Crossley (1995)]{cattle1995}
 Cattle, H. and Crossley, J.,
 Modelling Arctic Climate Change,
 {\em Phil. Trans. Roy. Soc. Lon. A}, {\bf 352}, 201--213, 1995.

\bibitem[Cox et al. (1999)]{cox1999}
 Cox PM, Betts RA, Bunton CB, Essery RLH, Rowntree PR, Smith
 J, The impact of new land surface physics on the GCM simulation
 of climate and climate sensitivity, {\it Clim Dyn}, {bf 15}, 183--203.

\bibitem[Cramer et al. (1999)]{cramer1999}
 Cramer W, Kicklighter DW, Bondeau A et al., Comparing global models of
 terrestrial net primary productivity (NPP): Overview and key results.
 {\it Global Change Biology}, {\bf 5} (Supplement 1), 1--15, 1999.

\bibitem[DeLucia et al. (2007)]{delucia2007}Delucia, E.H., Drake,
 J.E., Thomas, R.B. and Gozalez-Meler, M., Forest carbon use effciency:
 is respiration a constant  fraction of gross primary production?
 {\it Global Change Biology}, {\bf 13}, 1157--1167, 2007. 

\bibitem[Dewar (1997)]{dewar1997}
 Dewar, R. C., 1997, A simple model of light and water use efciency for pinus
 radiata, {\it Tree Physiology}, {\bf 17}, 259--265, 2007. 

\bibitem[Dommenget and Latif (2000)]{dommenget2000}
 Dommenget, D. and Latif, M.,
 Generation of SST anomalies in the midlatitudes,
 {\em Max-Planck-Institut Report}, {\bf 304}, 2000.

\bibitem[Eliassen et~al. (1970)]{eliassen1970}
 Eliassen, E., Machenhauer, B., and Rasmusson, E.,
 On a numerical method for integration of the hydrodynamical
 equations with a spectral representaion of the horizontal fields,
 {\em Inst. of Theor. Met., Univ. Copenhagen}, 1970.

\bibitem[Field et al. (1995)]{field1995}
 Field, C. B., J. T. Randerson, and C. M. Malmstrom,
 Global net primary production: Combining ecology and remote sensing,
 {\it Remote Sens. Environ.}, {\bf 51}, 74--88, 1995.

\bibitem[Gao et al. (2005)]{gao2005}
 Gao, F., C. B. Schaaf, A. H. Strahler, A. Roesch, W. Lucht,
 and R. Dickinson, 2005: MODIS bidirectional reflectance distribution
 function and albedo Climate Modeling Grid products and the variability
 of albedo for major global vegetation types, {\it J. Geophys. Res.},
 {\bf 110}, D01104, doi:10.1029/2004JD005190.

\bibitem[Garc\'{\i}a-Ojalvo and Sancho (1999)]{sancho}
 Garc\'{\i}a-Ojalvo, J. and  Sancho, J. M.,
 Noise in spatially extended systems,
 {\em Springer-Verlag, New-York}, 1999.

\bibitem[Gaspar (1988)]{gaspar1988}
 Gaspar, P.,
 Modeling the seasonal cycle of the upper ocean,
 {\em J. Phys. Oceanogr.}, {\bf 18}, 161--180, 1988.

\bibitem[Gordon et~al. (2000)]{gordon2000}
 Gordon, C. et~al.
 The simulation of SST, sea ice extents and ocean heat transports
 in a version of the Hadley Centre coupled model without flux adjustments,
 {\em Clim. Dyn.}, {\bf 16}, 147--168, 2000.

\bibitem[Hagemann et~al. (1999)]{hagemann1999}
 Hagemann, S., Botzet, M., D\"umenil, L. and Machenhauer, B.
 Derivation of global GCM boundary layer conditions from 1 km land use
 satellite data
 {\em MPI Report, Max Planck Institute for Meteorology, Hamburg}, {\bf 289}, 1999.\\
% {\small available at: http://www.mpimet.mpg.de/fileadmin/publikationen/Reports/max\_scirep\_289.pdf}


\bibitem[Hagemann (2002)]{hagemann2002}
 Hagemann, S., 2002
 An improved land surface parameter dataset for global and regional 
 climate models 
 {\em MPI Report, Max Planck Institute for Meteorology, Hamburg}, {\bf 336}, 2002.\\
% {\small available at: http://www.mpimet.mpg.de/fileadmin/publikationen/Reports/max\_scirep\_336.pdf}

\bibitem[Haltiner and Williams (1982)]{haltiner1982}
 Haltiner, G. J. and Williams, R. T.,
 Numerical Prediction and Dynamic Meteorology,
 {\em John Wiley and Sons, New York}, 1982.

\bibitem[Harvey (1989)]{harvey1989}
 Harvey, L. D. D., 1989, Effect of model structure on the response of
 terrestrial biosphere models to CO2 and temperature increases,
 {\it Global Biogeochemical Cycles}, {\bf 3(2)}, 137--153, 1989.

\bibitem[Hewitt (2000)]{hewitt2000}
 Hewitt, G.,
 The genetic legacy of the Quaternary ice ages,
 {\em Nature}, {\bf 405}, 907--913, 2000.

\bibitem[Hibler and Zhang (1993)]{hibler1993}
 Hibler, W. D. III and Zhang, J.
 Interannual and climatic characteristics of an ice ocean
 circulation model,
 {\em Springer-Verlag New York},
 NATO ASI Series Global and Environmental Change, 1993.

\bibitem[Hoskins and Simmons (1975)]{hoskins}
 Hoskins, B. J. and Simmons, A. J.,
 A multi-layer spectral method and the semi-implicit method,
 {\em Q. J. R. Meteorol. Soc.}, {\bf 101}, 637--655, 1975.

\bibitem[Houtekamer and Derome (1995)]{Houtekamer95}
 Houtekamer, P. L. and Derome, J., Methods for ensemble prediction,
 {\em Mon. Wea. Rev.}, {\bf 123}, 2181--2196, 1995.

\bibitem[Houtekamer et~al.(1996)]{Houtekamer} Houtekamer, P.L.,
 Lefaivre, L., Derome, J., Ritchie, H. and Mitchell, H., A system
 simulation approach to ensemble prediction, {\em Mon. Wea. Rev.},
 {\bf 124}, 1225--1242, 1996.

\bibitem[Karaca and M\"uller (1991)]{karaca1991}
 Karaca, M. and M\"uller, D.,
 Mixed-layer dynamics and buoyancy transports,
 {\em Tellus}, {\bf 43}, 350--365, 1991.

\bibitem[Kiehl et~al. (1996)]{kiehl1996}
 Kiehl, J. T., Hack, J. J., Bonan, iG. B., Boville, iB. A., Briegleb, B. P.,
 Williamson, D. L. and Rasch, P. J.,
 Description of the NCAR Community Climate Model (CCM3),
 {\em National Centre for Atmospheric Research}, 1996.

\bibitem[King and Turner (1997)]{king1997}
 King, J. C. and Turner, J.,
 Antarctic Meteorology and Climatology,
 {\em Cambridge University Press}, 1997.

\bibitem[Kleidon (2006)]{kleidon2006}
 Kleidon A., The climate sensitivity to human appropriation of vegetation
 productivity and its thermodynamic characterization,
 {\it Glob Planet Change}, {\bf 54}, 109--127, 2006.

\bibitem[Knorr (2000)]{knorr2000}
 Knorr, W., Annual and interannual CO2 exchange of the terrestrial
 biosphere: Process based simulations and uncertainties,
 {\it Global Ecol Biogeogr}, {bf 9}, 225-252, 2000.

\bibitem[Kraus (1967)]{kraus1967}
 Kraus, E. B. and Turner, J. S.,
 One-dimensional model of the seasonal thermocline II.
 The general theory and its consequences.
 {\em Tellus}, {\bf 19}, 98--105, 1967.

\bibitem[Lohmann and Gerdes (1998)]{lohmann1998}
 Lohmann, G. and Gerdes, R.,
 Sea Ice Effects on the Sensitivity of the Thermohaline,
 {\em J. Clim.}, {\bf 11}, 2789--2803, 1998.

\bibitem[Lorenzo and P\'{e}rez-Mu\~nuzuri (1999)]{lorenzo1999}
 Lorenzo, M. N. and P\'{e}rez-Mu\~nuzuri, V.,
 Colored noise-induced chaotic array synchronization,
 {\em Phys. Rev. E}, {\bf 60}, 2779--2787, 1999.

\bibitem[Lorenzo and P\'{e}rez-Mu\~nuzuri(2001)]{lorenzo2001}
 Lorenzo, M.N. and P\'{e}rez-Mu\~nuzuri, V.,  Influence of low
 intensity noise on assemblies of diffusively coupled chaotic
 cells, {\em Chaos}, {\em 11}, 371--376, 2001.

\bibitem[Lorenzo et~al.(2002)]{lorenzo2002}
 Lorenzo, M.N., Santos M.A. and P\'{e}rez-Mu\~nuzuri, V.,
 Spatiotemporal stochastic forcing effects in an ensemble
 consisting of arrays of diffusively coupled Lorenz cells, {\em
 submitted to Phys. Rev. E}, 2002.

\bibitem[Lunkeit(2001)]{lunkeit2001}
 Lunkeit, F., Synchronization experiments with an atmospheric
 global circulation model, {\em Chaos}, {\em 11}, 47--51, 2001.

\bibitem[McGuire et al. (1992)]{mcguire1992}
 McGuire, A. D., J. M. Melillo, L. M. Joyce, D. M. Kicklighter,
 A. L. Grace, B. Moore III, and C. J. Vorosmarty, Interactions
 between carbon and nitrogen dynamics in estimating net primary
 productivity for potential vegetation in North America,
 {\it Global Biogeochem. Cycles}, {\bf 6}, 101--124, 1992.

\bibitem[Molteni et~al.(1996)]{molteni} Molteni, F., Buizza, R.,
 Palmer T.N. and Petroliagis, T., The ECMWF ensemble prediction
 system: Methodology and validation, {\em Q.J.R. Meteorol. Soc.},
 {\em 122}, 73--120, 1996.

\bibitem[Monteith et al. (1989)]{monteith1989}
 Monteith, J.L., A.K.S. Huda, and D. Midya. 1989. RESCAP: a resource
 capture model for sorghum and pearl millet.  In {\it Modelling
 the Growth and Development of Sorghum and Pearl Millet},  Eds.
 S.M. Virmani, H.L.S. Tandon, and G. Alagarswamy.
 ICRISAT Research Bulletin 12, Patancheru, India, pp 30--34.

\bibitem[Morison and Gifford (1983)]{morison1983}
 Morison J.I.L. and Gifford R.M., Stomatal sensitivity to carbon dioxide
 and humidity, {\it Plant Physiology}, {bf 71}, 789--796, 1983.

\bibitem[Orszag (1970)]{orszag1970}
 Orszag, S. A.,
 Transform method for calculation of vector coupled sums,
 {\em J. Atmos. Sci.}, {\bf 27,} 890-895.

\bibitem[Parkinson and Washington (1979)]{parkinson1979}
 Parkinson, C. L. and Washington, W. M.,
 A large-scale numerical model of sea ice,
 {\em J. Geophys. Res.}, {\bf 84}, 311--337, 1979.

\bibitem[Phillips (1957)]{phillips1957}
 Phillips, N. A.,
 A coordinate system having some special advatages for
 numerical forecasting,
 {\em J. Meteorology}, {\bf 14}, 184--185, 1957.

\bibitem[Polley et al. (1993)]{polley1993}
 Polley, H.W., H.B. Johnson, B.D. Marino and H.S. Mayeux, Increase in C3
 plant water-use efficiency and biomass over Glacial to present CO2
 concentrations, {\it Nature}, {\bf 361}, 61--64, 1993.

\bibitem[Potter et al. (1993)]{potter1993}
 Potter CS, Randerson J, Field CB, Matson PA, Vitousek PM, Mooney HA,
 Klooster SA., Terrestrial ecosystem production: a process model based
 on global satellite and surface data,
 {\it Global Biogeochemical Cycles}, {\bf 7},811--841, 1993.

\bibitem[Rechid et al. (2008)]{rechid2008}
 Rechid, D., Hagemann, S., Jacob, D.,  
 Sensitivity of climate models to seasonal variability of 
 snow-free land surface albedo.
 {\em Theor Appl Climatol, DOI 10.1007/s00704-007-0371-8}, 2008

\bibitem[Rechid et al. (2009)]{rechid2009}
 Rechid, D., T. J. Raddatz, D. Jacob, Parameterization of snow-free land
 surface albedo as a function of vegetation phenology based on MODIS
 data and applied in climate modelling,
 {\it Theor Appl Climatol}, {\bf 95}, 245--255, 2009.

\bibitem[Roesch and Roeckner (2006)]{roesch2006}
 Roesch, A. and E. Roeckner, Assessment of Snow Cover and Surface Albedo
 in the ECHAM5 General Circulation Model, {\it Journal of Climate},
 {\bf 19}, 3828--3843, 2006.

\bibitem[Roesch et al. (2001)]{roesch2001}
 Roesch, A., M. Wild, H. Gilgen, A. Ohmura, A new snow cover fraction
 parametrization for the ECHAM4 GCM,
 {\it Climate Dynamics}, {\bf 17}, 933-946, 2001.   

\bibitem[Santos and Sancho(2001)]{santos} Santos, M.A. and Sancho,
 J.M., Front dynamics in the presence of spatiotemporal noises,
 {\em Phys. Rev. E}, {\em 64}, 016129(1)--016129(11), 2001.

\bibitem[Semtner (1976)]{semtner1976}
 Semtner, A. J. Jr.,
 A Model for the Thermodynamic Growth of Sea Ice in Numerical
 Investigations of Climate,
 {\em J. Physic. Oceanogr.}, {\bf 3}, 379--389, 1976.

\bibitem[Simmons et~al.(1978)]{simmons1978}
 Simmons, A. J., B. J. Hoskins, and D. M. Burridge, 1978:
 Stability of the semi-implicit method of time integration.
 {\em Mon. Wea. Rew.}, {\bf 106,} 405--412.

\bibitem[Simmons and Burridge (1981)]{simmons1981}
 Simmons, A. J., and D. M. Burridge, 1981:
 An Energy and Angular-Momentum Conserving Vertical Finite-Difference
 Scheme and Hybrid Vertical Coordinates.
 {\em Mon. Wea. Rew.}, {\bf 109,} 758--766.

\bibitem[Smith et~al.(1999)]{smith}
 Smith, L.A.,  Ziehmann, C. and  Fraedrich, K., Uncertainty
 dynamics and predictability in chaotic systems, {\em Q.J.R.
 Meteorol. Soc.}, {\em 125}, 2855--2886, 1999.

\bibitem[Tibaldi and Geleyn, (1981)]{tibaldi1981}
 Tibaldi, S., and J.-F. Geleyn, 
 The production of a new orography, land-sea mask and associated climatological surface
 fields for operational purposes
 {\em ECMWF Technical Memorandum}, {\bf  40} 1981.

\bibitem[Timmermann (2000)]{timmermann2000}
 Timmermann, R.,
 Wechselwirkungen zwischen Eis und Ozean im Weddelmeer,
 {\em University of Bremen}, 2000.

\bibitem[Toth and Kalnay(1993)]{toth} Toth, Z. and Kalnay, E.,
 Ensemble forecasting at NMC: The generation of perturbations, {\em
 Bull. Amer. Meteor. Soc.}, {\em 74}, 2317--2330, 1993.

\bibitem[Turnbull et al. (2002)]{turnbull2002}
 Turnbull, M.H., D. Whitehead, D. T. Tissue, W. S. F. Schuster,
 K. J. Brown, V. C. Engel, K. L. Griffin, Photosynthetic
 characteristics in canopies of Quercus rubra, Quercus prinus
 and Acer rubrum differ in response to soil water availability,
 {\it Oecologia}, {bf 130}, 515--524, 2002. 

\bibitem[UNESCO (1978)]{unesco1978}
 Eighth report of the joint panel on oceanographic tables and standards,
 {\em UNESCO Technical Papers in Marine Science}, {\bf 28}, 1978.

\bibitem[Whitaker and Lougue(1998)]{whitaker}
 Whitaker, J.S. and  Lougue, A.F., The relationship between
 ensemble spread and ensemble mean skill, {\em Mon. Wea. Rev.},
 {\em 126}, 3292--3302, 1998.

\bibitem[Wilks(1995)]{wilks} Wilks, D.S., {\it Statistical methods in the
 atmospheric sciences}, Academic Press, New-York, 1995.

\bibitem[Williamson et al. (2006)]{williamson2006}
 Williamson, M. S., T.M. Lenton, J.G. Shepherd, N.R. Edwards, An
 effcient numerical terrestrial scheme (ENTS) for Earth 
 system modelling,
 {\it Ecological Modelling}, {\bf 198}, 362--374, 2006.

\bibitem[Wong et al. (1979)]{wong1979}
 Wong, S.C., I. R. Cowan, and G. D. Farquhar, Stomatal conductance
 correlates with photosynthetic capacity,
 {\it Nature}, {\bf 282}, 424--426, 1979.

\bibitem[Yuan et al. (2007)]{yuan2007}
 Yuan, W. and coauthors,
 Global pattern of NPP to GPP ratio derived from MODIS data: effects of
 ecosystem type, geographical location
 and climate,  {\it Global Ecology and Biogeography},
 {\bf 18}, 280--290, 2007.

\bibitem[Zhang et al. (2009)]{zhang2009}
 Zhang, Y., M. Xu, H. Chen, J. Adams, 
 Deriving a light use effciency model from eddy covariance flux 

\end{thebibliography}
